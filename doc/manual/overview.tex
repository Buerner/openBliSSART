%
% $Id$
% vim:set expandtab tabstop=4 shiftwidth=4:
%


openBliSSART is a framework and toolbox for Blind Source Separation for Audio
Recognition Tasks. Main features include

\begin{itemize}
\item Component separation using non-negative matrix factorization (NMF)
  \cite{LeeSeung1999,Smaragdis2003,LeeSeung2001} and non-negative matrix deconvolution
  (NMD) \cite{Smaragdis2004}
\item Component classification:
\begin{itemize}
    \item Feature extraction from components
    \item Creation of response variables assigning audio components to classes
    \item Assembly of audio files for different
    classes, such as in drum beat separation \cite{Virtanen2005}
\end{itemize}
\item Supervised and unsupervised NMF feature extraction
\item Data export (ARFF \cite{Weka} and HTK \cite{HTKBook} formats)
\end{itemize}

\begin{leftbar}
In many places in this document and the applications, NMF and NMD are used
as synonyms. The reason is that mathematically NMF is a special case of NMD.
\end{leftbar}

The remainder of this manual is divided into three chapters. Section \ref{chapter:Tutorial} provides
a brief introductory tutorial on how to use openBliSSART for a typical blind
source separation task. Section \ref{chapter:RefManual} explains the data storage architecture and algorithmic concepts of openBliSSART in detail.

For detailed information about the classes in the openBliSSART framework, please consult the
corresponding HTML documentation.

